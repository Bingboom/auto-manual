% ==================================================
% components_base.tex
% 通用组件 / 通用排版规则(可复用)
% Parameter access: \csname HB<key>\endcsname
% ==================================================

% --- list base ---
\setlist[itemize]{
  leftmargin=\csname HBlist_leftmargin\endcsname,
  labelsep=\csname HBlist_labelsep\endcsname,
  itemsep=\csname HBlist_itemsep\endcsname,
  topsep=\csname HBlist_topsep\endcsname,
  parsep=\csname HBlist_parsep\endcsname,
  partopsep=\csname HBlist_partopsep\endcsname,
  label=\raisebox{\csname HBlist_bullet_raise\endcsname}{\csname HBlist_bullet_symbol\endcsname}
}

% --- table base: inner rules >= 0.2pt ---
\arrayrulewidth=\csname HBtable_inner_rule\endcsname
\setlength{\tabcolsep}{\csname HBtable_tabcolsep\endcsname}

% Outer border wrapper (>=0.4pt)
\newtcolorbox{tableframe}{
  enhanced,
  colback=white,
  colframe=LineK40,
  arc=\csname HBtable_outer_arc\endcsname,
  boxrule=\csname HBtable_outer_rule\endcsname,
  left=0mm,right=0mm,top=0mm,bottom=0mm,
}

\setcounter{secnumdepth}{0}

% NOTE box (K5% gray background)
\newtcolorbox{notebox}{
  enhanced,
  colback=BgK05,
  colframe=BgK05,
  arc=\csname HBnote_arc\endcsname,
  boxrule=0pt,
  left=\csname HBnote_pad_lr\endcsname,
  right=\csname HBnote_pad_lr\endcsname,
  top=\csname HBnote_pad_tb\endcsname,
  bottom=\csname HBnote_pad_tb\endcsname,
}

% Make rubric (subsubsection*) look like original: small + dark
\makeatletter
\renewcommand\subsubsection{\@startsection{subsubsection}{3}{0pt}%
  {\csname HBrubric_before\endcsname}{\csname HBrubric_after\endcsname}%
  {\bfseries\fontsize{\csname HBrubric_font_size\endcsname}{\csname HBrubric_font_leading\endcsname}\selectfont\color{black}}}
\makeatother
